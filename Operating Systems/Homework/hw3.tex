\documentclass{article}

\begin{document}

\title{Homework 3}
\author{Mitchel Fields}
\date{4/9/15}
\maketitle

\begin{enumerate}

\item It is safe if the processes continue in the order P2, P3, P4, P1.

\item Fragmentation:
\begin{enumerate}
\renewcommand{\theenumi}{\Alph{enumi}}
\item No. On average 1/2 a page is wasted for every process, causing internal fragmentation.

\item Yes. Pages are of finite predetermined size so programs are allocated consistent amounts of space, eliminating external fragmentation.
\end{enumerate}

\item To allow translation from program internal addresses (i.e. text line 26) to physical memory addresses (page number + offset)

\item TLB (translation lookaside buffer). The TLB contains the page to frame mapping of the most frequently accessed pages to improve performance when reaccessing those pages.

\item 
\begin{enumerate}
\renewcommand{\theenumi}{\Alph{enumi}}
\item 4 for page and 2 for offset

\item $31 \rightarrow p = 0, d = 32 \rightarrow p1 = f3 \rightarrow (3*32)-1+32 = (4*32)-1 = 127$

\end{enumerate}

\item Segmented Paging:
\begin{enumerate}
\renewcommand{\theenumi}{\Alph{enumi}}
\item Virtual addresses are split into variably sized segments which are then mapped to locations in constant size pages throughout system memory.
\item Two levels of code sharing: at the page and segment level. Easy to allocate memory of various sizes.
\item More resource consuming to map a virtual address to a physical one through multiple tables.
\end{enumerate}



\item Spatial and Temporal Locality
\begin{enumerate}
\renewcommand{\theenumi}{\Alph{enumi}}
\item Temporal locality of reference is the idea that recently accessed information will likely be accessed in the future.
\item Spatial locality of reference is the idea that information near recently accessed information is likely to be accessed in the future.
\item Locality of reference helps guide what information is most crucial to store in memory and cache space based on reasonable predictions of how likely it is to be accessed in the near future based on whether it was recently accessed and whether currently processed information is close to it spatially in the process's virtual memory.
\end{enumerate}


\end{enumerate}

\end{document}