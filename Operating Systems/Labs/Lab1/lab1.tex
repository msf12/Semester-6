\documentclass{article}

\begin{document}

\paragraph{\Large Activity 1}\mbox{}\\
\begin{enumerate}
	\item UNIVAC:
 One of the first commercially available computers in the US. It used a very rudimentary and simple OS to handle IO.
	\item Whirlwind:
 The Whirlwind was one of the first digital electronic computers to do real-time computations based on user inputs via a console, and had a minimal OS to facilitate that.
	\item IBM 7090:
 The transistorized version of the IBM 709 and had an OS called IBSYS that supported multiple programming languages and featured data IO using data channels connected to tape drives.
	\item GE-645:
 Used the Multics OS and supported virtual memory and multiprogramming and, as a result, was often used in multiple user settings.
	\item Data General Eclipse MV/8000: The first of Data General's 32-bit minicomputer. It ran the AOS operating system, which had 8 distinct rings of privilege to protect data.
	\item PDP-11: A competitor to the Data General computers that used one bus for memory access and IO, supported 4 levels of interrupt vectors, and was designed for easier mass-production.

\end{enumerate}

\paragraph{\Large Activity 2}\mbox{}\\
\indent Uniprogramming is sequential execution of programs whereas multiprogramming is sequential execution of parts of programs, which are rapidly switched between to give the appearance of parallelism and to allow the CPU to continue doing work while a program halts for IO. Both can be implemented on uniprocessors and multiprocessors.
	
\end{document}