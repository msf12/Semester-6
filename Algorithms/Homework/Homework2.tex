\documentclass{article}
\usepackage{listings}

\begin{document}
\title{Homework 2}
\date{}
\maketitle

\paragraph{\Large 1. Question 1.4.4}\mbox{}\\
Develop a table like the one on page 181 for TwoSum.\\
\begin{lstlisting}[language=Java]
public class TwoSum {
    public static int count(int[] a) {
        /* BLOCK A */
        int N = a.length;
        int cnt = 0;
        for (int i = 0; i < N; /* BLOCK B */ i++)
            for (int j = i+1; j < N; /* BLOCK C */ j++)
                if (a[i] + a[j] + a[k] == 0)
                    /* BLOCK D */ cnt++;
        /* BLOCK A CONTINUED */
        return cnt;
    }
    ...
}
\end{lstlisting}
\begin{tabular}{c c c c}
statement block & time in seconds & frequency & total time \\ \hline
D & $t_0$ & $x$ (depends on input) & $t_0x$ \\
C & $t_1$ & $N^2/2+N/2$ & $t_1(N^2/2-N/2)$\\
B & $t_2$ & $N$ & $t_2N$ \\
A & $t_3$ & 1 & $t_3$ \\
\end{tabular}


\paragraph{\Large 2. Fall 2010 Midterm Question 1d}\mbox{}\\
Consider the following Java data type definition for a 2-3 tree, where the nested class
Node represents either a 2-node or a 3-node.
\begin{lstlisting}[language=Java]
public class TwoThreeTree<Key extends Comparable<Key>, Value> {
    private Node root;
    private class Node {
        private int count; // subtree count
        private Key key1, key2; // the one or two keys
        private Value value1, value2; // the one or two values
        private Node left, middle, right; // the two or three subtrees
    }
    ...
}
\end{lstlisting}
How much memory (in bytes) does each Node object consume?

Node - 16 bytes of overhead + 8 bytes of extra overhead because Node is nested\\
count - 8 bytes (4 int bytes + 4 bytes of padding)\\
key1 and key2 - 8 bytes each (references)\\
value1 and value 2 - 8 bytes each (references)\\
left, middle, right - 8 bytes each (references)\\

$16 + 8 + 8 + 8*2 + 8*2 + 8*3 = 88$ bytes per Node\\

\paragraph{\Large 3. Fall 2011 Midterm Question 2}\mbox{}\\
Suppose that you collect the following timing data for a program as a function of the input
size $N$.\\\\
\begin{tabular}{c | c}
N & time \\ \hline
125 & 0.03 sec \\
1,000 & 1.00 sec \\
8,000 & 32.00 sec \\
64,000 & 1,024.00 sec \\
512,000 & 32,768.00 sec
\end{tabular}\\\\
Estimate the running time of the program (in seconds) as a function of $N$ and use tilde
notation to simplify your answer.\\
\textit{Hint}: recall that $\log_b a = \lg a / \lg b$:\\

\begin{tabular}{c | c}
Ratio $8N/N$ & Ratio T$(8N)/$T$(N)$ \\ \hline
8 & 33.33333... \\
8 & 32 \\
8 & 32 \\
8 & 32 \\
8 & 32 \\
\end{tabular}\\

\noindent $\lg 32/\lg 8 = \log_8 32 = 5/3 \\
b = 5/3 \\
a*1000^{5/3}=1.00\\
a=1/100000
T(N) = 1/100000N^{5/3} \sim N^{5/3}$

\end{document}