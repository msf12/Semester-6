\documentclass{article}

\usepackage{qtree}

\begin{document}
\title{Homework 8}
\date{}
\maketitle


% 1. What is the best case BST height? Worst case?
% 2. If shuffling guarantees log N tree height (probabilistically), why don't we simply shuffle our input data before building our BST based symbol table to avoid worst case behavior?
% 3. Textbook 3.2.3, but give only two orderings.
% 4. Textbook 3.2.4

\paragraph{\Large 1. Study Guide Question}\mbox{}\\
What is the best case BST height? Worst case?\\

Best case height for a BST is about $\lg N$ and worst case height is $N$.

\paragraph{\Large 2. Study Guide Question}\mbox{}\\
If shuffling guarantees log N tree height (probabilistically), why don't we simply shuffle our input data before building our BST based symbol table to avoid worst case behavior?\\

Shuffling the input data would affect what values each key has at the end of reading the input. For example, if, in the data set, key E is given the value 6 and then later given the value 3, shuffling the data makes the result of constructing the tree indeterminate with respect to the value of E.

\paragraph{\Large 3. Question 3.2.3, but give only two orderings}\mbox{}\\
Give five orderings of the keys  A X C S E R H that, when inserted into an initially empty BST, produce the \textit{best case} tree.

\Tree [ .H [ .E A C ] [ .S R X ] ]

H E A C S R X\\

H E S A C R X

\paragraph{\Large 4. Question 3.2.4}\mbox{}\\
Suppose that a certain BST has keys that are integers between 1 and 10, and we search for 5. Which sequence below \textit{cannot} be the sequence of keys examined?
\begin{enumerate}
\renewcommand{\theenumi}{\Alph{enumi}}
	\item 10, 9, 8, 7, 6, 5
	\item 4, 10, 8, 6, 5
	\item 1, 10, 2, 9, 3, 8, 4, 7, 6, 5
	\item 2, 7, 3, 8, 4, 5
	\item 1, 2, 10, 4, 8, 5
\end{enumerate}

D is impossible. A BST cannot have a subtree of 7 with a number greater and less than 7 (8 and 4, in this case).

\end{document}