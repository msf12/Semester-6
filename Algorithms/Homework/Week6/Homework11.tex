\documentclass{article}

\usepackage{listings}

\begin{document}
\title{Homework 11}
\date{}
\maketitle

% 1. Textbook 3.4.5
% 2. Fall 2010 Midterm, #5, Fall 2012 Midterm, #6a

\paragraph{\Large 1. Question 3.4.5}\mbox{}\\
Is the following implementation of hashcode() legal?
\begin{lstlisting}[language=Java]
public int hashcode()
{ return 17; }
\end{lstlisting}

This implementation of hashcode is a legal implementation. It maps objects to an integer value. However, it is a very poor implementation as it maps every object to the same integer, meaning that the hashed objects will not be uniformly spread.

\paragraph{\Large 2. One of the following: Fall 2010 Midterm Question 5 or Fall 2012 Midterm Question 6a}\mbox{}\\
Suppose that the following keys are inserted in the order\\
A B C D E F G\\
into an initially empty linear-probing hash table of size 7, using the following hash function:\\\\
\begin{tabular}{c | c}

key & hash(key, 7)\\ \hline
A & 3 \\
B & 1 \\
C & 4 \\
D & 1 \\
E & 5 \\
F & 2 \\
G & 5 \\
\end{tabular}\\\\
What is the result of the linear-probing array? Assume that the array size is fixed and does not double.\\

\begin{tabular}{c | c | c | c | c | c | c }
0 & 1 & 2 & 3 & 4 & 5 & 6\\ \hline
G & B & D & A & C & E & F
\end{tabular}

\end{document}